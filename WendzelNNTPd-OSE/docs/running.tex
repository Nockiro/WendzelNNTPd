\chapter{Starting and Running WendzelNNTPd}\index{Starting WendzelNNTPd}

\section{Starting the Service}\index{Starting WendzelNNTPd}

Once your WendzelNNTPd installation has been configured, you can run the server (in the default case you need superuser access to do that since this is required to bind WendzelNNTPd to the default NNTP port 119) by starting {\bf /usr/local/sbin/wendzelnntpd}.

\begin{verbatim}
$ /usr/local/sbin/wendzelnntpd 
WendzelNNTPd-OSE (Open Source Edition): version 2.0.7 'Berlin'  - (Oct
26 2015 14:10:20 #2544) is ready.
\end{verbatim}

{\bf Note (Daemon Mode):} If you want to run WendzelNNTPd as a background daemon process on *nix-like operating systems, you should use the parameter {\bf -d}.

\section{Stopping and Restarting the Service}\index{Stopping WendzelNNTPd}\index{Restarting WendzelNNTPd}

The server can be stopped by killing its process:

\begin{verbatim}
$ pkill wendzelnntpd
\end{verbatim}

The server has a handler for the kill signal that allows to safely shutdown using {\bf pkill} or {\bf kill}.

To restart the service, kill and start the service.

\subsection{Automating Start, Stop, and Restart}

The script {\bf init.d\_script} in the directory {\it scripts/startup} of the tarball can be used to start, restart, and stop WendzelNNTPd. It is a standard init.d script for Linux operating systems that can usually be copied to {\it /etc/init.d} (it must be executable).

\begin{verbatim}
$ cp scripts/startup/init.d_script /etc/init.d/wendzelnntpd
$ chmod +x /etc/init.d/wendzelnntpd
\end{verbatim}

{\bf Note:} Please note that some operating systems use different directories than {\it /etc/init.d} or other startup script formats. In such cases, the script works nevertheless but should simply be installed to {\it /usr/local/sbin} instead.


To start, stop, and restart WendzelNNTPd, the following commands can be used afterwards:

\begin{verbatim}
$ /etc/init.d/wendzelnntpd start
Starting WendzelNNTPd ... done.
WendzelNNTPd-OSE (Open Source Edition): version 2.0.7 'Berlin'  - (Oct
26 2015 14:10:20 #2544) is ready.

$ /etc/init.d/wendzelnntpd restart
Stopping WendzelNNTPd ... done.
Starting WendzelNNTPd ... done.
WendzelNNTPd-OSE (Open Source Edition): version 2.0.7 'Berlin'  - (Oct
26 2015 14:10:20 #2544) is ready.

$ /etc/init.d/wendzelnntpd stop
Stopping WendzelNNTPd ... done.
\end{verbatim}



\section{Administration Tool `wendzelnntpadm'}

Use the command line tool {\bf wendzelnntpadm} to configure user, role and newsgroup settings of your WendzelNNTPd installation. To get an overview of supported commands, run ``wendzelnntpadm help'':

\begin{verbatim}
$ wendzelnntpadm help
usage: wendzelnntpd <command> [parameters]
*** Newsgroup Administration:
 <listgroups>
 <addgroup | modgroup> <newsgroup> <posting-allowed-flag (y/n)>
 <delgroup> <newsgroup>
*** User Administration:
 <listusers>
 <adduser> <username> [<password>]
 <deluser> <username>
*** ACL (Access Control List) Administration:
 <listacl>
 <addacluser | delacluser> <username> <newsgroup>
 <addaclrole | delaclrole> <role>
 <rolegroupconnect | rolegroupunconnect> <role> <newsgroup>
 <roleuserconnect | roleuserunconnect> <role> <username>
\end{verbatim}

\section{Creating/Listing/Deleting Newsgroups}

You can either list, create or delete newsgroups using {\bf wendzelnntpadm}.

\subsection{Listing existing newsgroups}

\begin{verbatim}
$ wendzelnntpadm listgroups
Newsgroup, Low-, High-Value, Posting-Flag
-----------------------------------------
alt.test 10 1 y
mgmt.talk 1 1 y
secret.project-x 20 1 y
done.
\end{verbatim}

\subsection{Creating a new newsgroup}

To create a new newsgroup run the following command:

\begin{verbatim}
$ wendzelnntpadm addgroup my.cool.group y
Newsgroup my.cool.group does not exist. Creating new group.
done.
\end{verbatim}

You can also change the ``posting allowed'' flag of a newsgroup but this takes currently no effect since WendzelNNTPd handles all newsgroups as newsgroups with posting allowed.

\begin{verbatim}
$ wendzelnntpadm modgroup my.cool.group y
Newsgroup my.cool.group exists: okay.
done.
$ wendzelnntpadm modgroup my.cool.group n
Newsgroup my.cool.group exists: okay.
done.
\end{verbatim}

\subsection{Deleting a newsgroup}

\begin{verbatim}
$ wendzelnntpadm delgroup my.cool.group
Newsgroup my.cool.group exists: okay.
Clearing association class ... done
Clearing ACL associations of newsgroup my.cool.group... done
Clearing ACL role associations of newsgroup my.cool.group... done
Deleting newsgroup my.cool.group itself ... done
Cleanup: Deleting postings that do not belong to an existing newsgroup ... done
done.
\end{verbatim}

\section{User Accounts Administration}

The easiest way to give only some people access to your server is to create user accounts (please make sure you activated authentication in your configuration file). You can add, delete and list all users.

\subsection{Listing Users (and Passwords)}

This command always prints you the password of a user too.

\begin{verbatim}
$ wendzelnntpadm listusers
Username, Password
------------------
developer1, wegerhgrhtrthjtzj
developer2, erghnrehhnht
manager1, wegergergrhth
manager2, thnthnrothnht
swendzel, lalalegergreg
swendzel2, 94j5z5jh5th
swendzel3, lalalalala
swendzel4, wegwegwegwegweg
done.
\end{verbatim}

\subsection{Creating a new user}

You can either enter the password as additional parameter (usefull for scripts that create users automatically) ...

\begin{verbatim}
$ wendzelnntpadm adduser UserName HisPassWord
User UserName does currently not exist: okay.
done.
\end{verbatim}

... or you can type it using the prompt (in this case the input is shadowed):

\begin{verbatim}
$ wendzelnntpadm adduser UserName2
Enter new password for this user (max. 100 chars):
User UserName2 does currently not exist: okay.
done.
\end{verbatim}

{\bf Please Note:} A password must include at least 8 characters and may not include more than 100 characters.

\subsection{Deleting an existing user}

\begin{verbatim}
$ wendzelnntpadm deluser UserName2
User UserName2 exists: okay.
Clearing ACL associations of user UserName2... done
Clearing ACL role associations of user UserName2... done
Deleting user UserName2 from database ... done
done.
\end{verbatim}

\section{Access Control List Administration (just in case, when the standard NNTP authentication is not enough)}

Welcome to the advanced part of WendzelNNTPd. WendzelNNTPd includes a powerful role based access control system. You can either only use normal access control lists where you can configure which user will have access to which newsgroup. Or you can use the advanced role system: You can add users to roles (e.g. the user ``boss99'' to the role ``management'') and give a role access to a group (e.g. role ``management'' shall have access to ``discuss.management'').

{\bf Note:} Please note that you must activate the ACL feature in your configuration file to use it.

{\bf Note:} To see *ALL* data related to the ACL subsystem of your WendzelNNTPd installation, simply use ``wendzelnntpadm listacl''.

\subsection{Invisible Newsgroups}

WendzelNNTPd includes a feature called ``Invisible Newsgroups'' which means that a user without access to a newsgroup will neither see the newsgroup in the list of newsgroups, nor will he be able to post to such a newsgroup or will be able to read it.

\subsection{Simple Access Control}

We should start with the simple access control where you can define which user should have access to which newsgroup.

\subsubsection{Giving a user access to a newsgroup}

\begin{verbatim}
$ wendzelnntpadm addacluser swendzel alt.test
User swendzel exists: okay.
Newsgroup alt.test exists: okay.
done.
$ wendzelnntpadm listacl
List of roles in database:
Roles
-----

Connections between users and roles:
Role, User
----------

Username, Has access to group
-----------------------------
swendzel, alt.test

Role, Has access to group
-------------------------
done.
\end{verbatim}

\subsubsection{Removing a user's access to a newsgroup}

\begin{verbatim}
$ wendzelnntpadm delacluser swendzel alt.test
User swendzel exists: okay.
Newsgroup alt.test exists: okay.
done.
\end{verbatim}

\subsection{Adding and Removing ACL Roles}

If you have many users, some of them should have access to the same newsgroup (e.g. the developers of a new system should have access to the development newsgroup of the system). With roles you do not have to give every user explicit access to such a group. Instead you add the users to a role and give the role access to the group. (One advantage is that you can easily give the complete role access to another group with only one command instead of adding each of its users to the list of people who have access to the new group).

In the following examples, we give the users ``developer1'', ``developer2'', and ``developer3'' access to the development role of ``project-x'' and connect their role to the newsgroups ``project-x.discussion'' and ``project-x.support''. To do so, we create the three users and the two newsgroups first:

\begin{verbatim}
$ wendzelnntpadm adduser developer1
Enter new password for this user (max. 100 chars):
User developer1 does currently not exist: okay.
done.
$ wendzelnntpadm adduser developer2
Enter new password for this user (max. 100 chars):
User developer2 does currently not exist: okay.
done.
$ wendzelnntpadm adduser developer3
Enter new password for this user (max. 100 chars):
User developer3 does currently not exist: okay.
done.

$ wendzelnntpadm addgroup project-x.discussion y
Newsgroup project-x.discussion does not exist. Creating new group.
done.
$ wendzelnntpadm addgroup project-x.support y
Newsgroup project-x.support does not exist. Creating new group.
done.
\end{verbatim}

\subsubsection{Creating a Role}

Before you can add users to a role and before you can connect a role to a newsgroup, you have to create a role (you have to choose an ASCII name for it). In this example, the group is called ``project-x''.

\begin{verbatim}
$ wendzelnntpadm addaclrole project-x
Role project-x does not exists: okay.
done.
\end{verbatim}

\subsubsection{Deleting a Role}

You can delete a role by using ``delaclrole'' instead of ``addaclrole'' like in the example above.

\subsection{Connecting and Disconnecting Users with/from Roles}

To add (connect) or remove (disconnect) a user to/from a role, you need to use the admin tool too.

\subsubsection{Connecting a User with a Role}

The second parameter (``project-x'') is the role name and the third parameter (``developer1'') is the username. Here we add our 3 developer users from the example above to the group project-x:

\begin{verbatim}
$ wendzelnntpadm roleuserconnect project-x developer1
Role project-x exists: okay.
User developer1 exists: okay.
Connecting role project-x with user developer1 ... done
done.
$ wendzelnntpadm roleuserconnect project-x developer2
Role project-x exists: okay.
User developer2 exists: okay.
Connecting role project-x with user developer2 ... done
done.
$ wendzelnntpadm roleuserconnect project-x developer3
Role project-x exists: okay.
User developer3 exists: okay.
Connecting role project-x with user developer3 ... done
done.
\end{verbatim}

\subsubsection{Disconnecting a User from a Role}

\begin{verbatim}
$ wendzelnntpadm roleuserconnect project-x developer1
Role project-x exists: okay.
User developer1 exists: okay.
Connecting role project-x with user developer1 ... done
done.
\end{verbatim}

\subsection{Connecting and Disconnecting Roles with/from Newsgroups}

Even if a role is connected with a set of users, it is still useless until you connect the role with a newsgroup.

\subsubsection{Connecting a Role with a Newsgroup}

To connect a role with a newsgroup, we have to use the command line tool for a last time (the 2nd parameter is the role, and the 3rd parameter is the name of the newsgroup). Here we connect our ``project-x'' role to the two newsgroups ``project-x.discussion'' and ``project-x.support'':

\begin{verbatim}
$ wendzelnntpadm rolegroupconnect project-x project-x.discussion
Role project-x exists: okay.
Newsgroup project-x.discussion exists: okay.
Connecting role project-x with newsgroup project-x.discussion ... done
done.
$ wendzelnntpadm rolegroupconnect project-x project-x.support
Role project-x exists: okay.
Newsgroup project-x.support exists: okay.
Connecting role project-x with newsgroup project-x.support ... done
done.
\end{verbatim}

\subsubsection{Disconnecting a Role from a Newsgroup}

This is done like in the example above but you have to use the command ``rolegroup{\bf un}connect'' instead of ``rolegroupconnect''.

\subsection{Listing your whole ACL configuration again}

Like mentioned before, you can use the command ``listacl'' to list your whole ACL configuration (except the list of users that will be listed by the command ``listusers'').

\begin{verbatim}
$ wendzelnntpadm listacl
List of roles in database:
Roles
-----
project-x

Connections between users and roles:
Role, User
----------
project-x, developer1
project-x, developer2
project-x, developer3

Username, Has access to group
-----------------------------
swendzel, alt.test

Role, Has access to group
-------------------------
project-x, project-x.discussion
project-x, project-x.support
done.
\end{verbatim}

\subsubsection{Saving time}

As mentioned above, you as the maintainer of the usenet server can safe time by using roles. If you add a new developer to the system, and the developer should have access to the two groups ``project-x.discussion'' and ``project-x.support'', you do not have to assign the user to both (or even more) groups. Instead, you just add the user to the role ``project-x'' that is already connected to both groups.

If you want to give all developers access to the group ``project-x.news'', you also do not have to connect every developer with the project. Instead, you just connect the rule with the newsgroup, what is one command instead of $n$ commands. Of course, this time saving concept also works if you want to delete a user.

\section{Hardening}

Besides the already mentioned authentification, ACL and RBAC features, the security of the server can be improved by putting WendzelNNTPd in a chroot environment or letting it run under an unpriviledged user account (the user then needs write access to /var/spool/news/wendzelnntpd and read access to (/usr/local)/etc/wendzelnntpd.conf). An unpriviledged user under Unix-like systems is also not able to create a listen socket on the default NNTP port (119) since all ports up to 1023 are reserved. This means that the server should use a port >= 1024 if it is started by a non-root user. Note: Some Unix systems may have another priviledged port configuration. 

Please also note that WendzelNNTPd can be easily identified due to its welcoming `banner' (desired code `200' message of NNTP). Tools such as \texttt{nmap} provide rules to identify WendzelNNTPd and its version this way. Theoretically, this could be changed by a slight code modification (welcome message, HELP output and other components that make the server identifiable). However, I do not recommend this as it is just a form of `security by obscurity'.

