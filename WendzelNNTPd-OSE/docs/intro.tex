\chapter{Introduction}\index{Introduction}

WendzelNNTPd is a tiny but easy to use Usenet server (NNTP server) for Linux, *nix and BSD. The server is written in C. For security reasons, it is compiled with stack smashing protection by default, if your compiler supports that feature.

\section{Features}

\subsection{License}

WendzelNNTPd uses the GPLv3 license.

\subsection{Database Abstraction Layer}

WendzelNNTPd contains a database abstraction layer. Currently supported database systems are SQlite3 and MySQL. New databases can be easily added.

\subsection{Security}

WendzelNNTPd contains different security features, the most important features are probably Access Control Lists (ACLs) and the Role Based Access Control (RBAC) system. ACL and RBAC are described in a own chapter. WendzelNNTPd is probably the first Usenet server with support for RBAC.

Another feature are so-called ``invisible newsgroups'': If access control is activated, a user without permission to access the newsgroup will not be able to see the existence of the newsgroup. In case he knows about the existence of the newsgroup nevertheless, he will not be able to post to or read from the newsgroup.

\subsection{Auto-prevention of double-postings}

In case a user sends a post to two equal newsgroups within one post command's ``Newsgroups:'' header tag, the server will add it only once to save memory on the server and the time of the readers.

\subsection{IPv6}

WendzelNNTPd supports IPv6. The server can listen on multiple IP addresses as well as multiple ports.

\subsection{Why this is not a perfect Usenet server}

WendzelNNTPd does not implement all NNTP commands, but the (most) important ones. Another problem is that the regular expression library used is not 100\% compatible with the NNTP matching in commands like ``XGTITLE''. A 3rd aspect is that WendzelNNTPd cannot share messages with other NNTP servers. Finally, WendzelNNTPd lacks support for encrypted connections.

\section{Contribute}

See the ``CONTRIBUTE'' file in the tarball.

\section{History}

The project started in 2004 under the name Xyria:cdpNNTPd, as part of the Xyria project that also contained a fast DNS server, called Xyria:DNSd. In 2007, I renamed it to WendzelNNTPd and stopped development of Xyria:DNSd. Version 1.0.0 was released in 2007, version 2.0.0 in 2011. Since then I have primarly fixed reported bugs and added minor features to keep the software alive, which is also planned for the next ten years. A detailed history can be found in the ``HISTORY'' file in the tarball.





